\documentclass{article}


\usepackage{titlesec}
\usepackage{titling}
\usepackage[margin=1.2in]{geometry}
\usepackage{enumitem}\setlist[description]{font=\textendash\enskip\scshape\bfseries,left=1cm}
\usepackage{xcolor}
\usepackage{hyperref}

\hypersetup{colorlinks=true, linkcolor=blue, filecolor=magenta, urlcolor=red}

\titleformat{\section}
{\huge\bfseries}
{}
{0em}
{}[\titlerule]

\titlespacing{\section}
{0pt}
{.6cm}
{.5cm}
 
\titleformat{\subsection}
{\bfseries\Large\color{red!50!black}}
{}
{.3em}
{}

\titleformat{\subsubsection}[runin]
{\bfseries}
{}
{1em}
{}

\renewcommand{\maketitle}
{\begin{center}
{\huge\bfseries
\theauthor}

\vspace{.5em}

GitHub: \url{https://github.com/fermezz}\\LinkedIn: \url{https://linkedin.com/in/fermezz}\\Email: \url{fermezz@gmail.com}

\end{center}}

\begin{document}


\title{Curriculum Vitae}
\author{Fernando Mezzabotta Rey}

\maketitle

\section{Habilidades Técnicas}
\subsection{Lenguajes de Programación}
\subsubsection{Python}
  \begin{description}
    \item He trabajado extensamente con Python, desde scripting hasta desarrollar servicios completos dentro del paradigma orientado a objetos.
  \end{description}
\subsubsection{Java}
  \begin{description}
    \item Conozco Java y sus principios pero nunca trabajé profesionalmente con él, más allá de hacer proyectos en la universidad.
  \end{description}
\subsubsection{Rust}
  \begin{description}
    \item ¡Me gusta mucho! Me gusta jugar con nuevas tecnologías y aprender todo el tiempo e incluso me gustaría trabajar profesionalmente algún día con Rust. Hice una pequeña y abandonada herramienta llamada jenr: \url{https://github.com/fermezz/jenr}.
  \end{description}
\subsection{CI/CD}
\subsubsection{Jenkins}
  \begin{description}
    \item Worked extensively with it. Setting up several regular-jobs as well as many pipelines, in combination with many of its plugins.
    \item He trabajado mucho con Jenkins, tanto como con sus "jobs" regulares como con Jenkins Pipelines, construyendo sistemas de CI complejos.
  \end{description}
\subsubsection{Travis CI}
  \begin{description}
    \item He trabajado con cosas bastante simples en Travis, principalmente para proyectos de código abierto.
  \end{description}
\subsection{Práticas de testeo}
\subsubsection{Test Driven Development [TDD]}
\subsection{Metodologías ágiles}
\subsubsection{Scrum}
\section{Experiencia Laboral}
\subsection{Eventbrite – Ingeniero de Software}
\subsubsection{Engineering Development Academy (EDA)}
  \begin{description}
    \item Como Ingeniero junior, participé en la academia de desarrollo que tiene Eventbrite para introducir ingenieros recién graduados o gente sin tanta experiencia en el mercado laboral.
  \end{description}
\subsubsection{Eventbrite Music}
  \begin{description}
    \item Participé desarrollando varias funcionalidades para el producto de Eventbrite llamado Eventbrite Music.
    \item Participé en construir y mantener la funcionalidad de buscar artistas para que los organizadores de eventos pudieran buscar a través de \textasciitilde5M de artistas y vincularlos a sus eventos.
    \item Junto con mi equipo documenté, diseñé, implementé y ayudé a mantener endpoints públicos y privados de nuestra API no-tan RESTful.
    \item Participé junto a mi equipo en ser dueños de uno de los tantos servicios remotos de nuestra arquitectura distribuída, el cual tiene que responder hasta a \textasciitilde200 requests por segundo.
    \item Monitoreé los servicios mencionados a través de herramientas como Datadog o Splunk, entre otras.
  \end{description}
\subsubsection{DevTools [Trabajo actual]}
  \begin{description}
    \item I'm participating on building an maintaining our CI/CD process with mainly Jenkins.
    \item Como parte del equipo de DevTools participé en construir y mantener nuestro proceso de CI para \textasciitilde300 personas.
    \item Ayudo a mantener y soportar la herramienta basada en Kubernetes que usamos para nuestro ambiente de desarrollo.
    \item Participo en matener y mejorar las imágenes de Docker que las aplicaciones usan en la herramienta antes mencionada.
  \end{description}
\subsubsection{Deuda técnica [Extra]}
  \begin{description}
    \item Participo en varios proyectos que tienen como objetivo migrar nuestros más de 100 proyectos usando Python de Python 2 a Python 3 y seguir manteniendo la versión de Python actualizada en el tiempo.
    \item Participated on deprecating a built-in-house, very old ORM for Python in favor of the Django ORM.
    \item Participé en deprecar un ORM para Python construído en Eventbrite para usar el ORM del proyecto de código abierto mantenido por la comunidad de Django.
    \item Participé en un esfuerzo que abarca a toda la compañía de romper un monolítico gigante en servicios remotos.
  \end{description}
\section{Educación Formal}
\subsection{Universidad Tecnológica Nacional – FRM}
\subsubsection{Ingeniería en Sistemas – Una materia pendiente}
\section{Sobre la Cultura}
  \begin{description}
    \item Me gusta tener proyectos paralelos de vez en cuando, además de las tareas específicas de mi posición. Investigar cosas nuevas me mantiene motivado y, por lo tanto, concentrado.
    \item ¡El trabajo en equipo es super importante! Busco llevarme bien con tus compañeros de trabajo y mantener una buena relación con mis mánagers porque lleva a mejores resultados.
  \end{description}
\section{Lenguajes}
\subsection{Inglés}
\subsubsection{Lectura – Avanzado}
\subsubsection{Escritura – Avanzado}


\end{document}
